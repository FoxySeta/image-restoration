\documentclass[a4paper]{article}
\usepackage[italian]{babel}
\usepackage{epigraph}
\usepackage{hhline}
% plotting
\usepackage{pgf}
\usepackage{xcolor}

\definecolor{bg}{RGB}{255,249,227}

\title{
  Ricostruzione di immagini \\ \large Relazione del progetto per
  l'insegnamento di Calcolo numerico
}
\author{
  R. Gianmaria,
  L. Tagliavini,
  S. Volpe
}

\date{
	Universit\`a di Bologna \\
  \today
}

\begin{document}

\maketitle
\thispagestyle{empty}

\pagebreak
\tableofcontents
\pagebreak

\epigraph{Lo scopo del calcolo è la conoscenza, non i numeri.}
{\textit{R. W. Hamming}}
\epigraph{... ma per lo studente, i numeri sono spesso la via migliore verso la
conoscenza}
{\textit{A. Ralston}}

\section{Introduzione}
La rappresentabilità delle immagini digitali in forma matriciale fa sì che la
loro manipolazione possa trarre beneficio dagli strumenti del calcolo numerico:
un problema di interesse pratico è la ricostruzione di tali immagini quando
degradate. Nell'esperimento che segue, ci proponiamo di effettuare un'analisi
compararata di esempi di implementazioni dei metodi risolutivi più noti per la
risoluzione di questo tipo di problemi. Assumeremo, fra le conoscenze pregresse
della lettrice e del lettore, il modello di formazione-registrazione di
un'immagine digitale e i principali metodi di regolarizzazione.

\section{Strumenti}
Tutte le procedure necessarie all'esperimento sono state implementate in Python
3. In particolare, viene fatto uso dei pacchetti:
\begin{itemize}
  \item \verb!matplotlib! per la creazione di grafici;
  \item \verb!numpy! per l'uso di matrici;
  \item \verb!scipy! per risolvere i problemi di ottimizzazione;
  \item \verb!skimage! per il calcolo delle metriche delle immagini.
\end{itemize}

\section{Procedimento}

\subsection{Generazione dell'insieme dei dati}
Sono state preparate otto immagini (da \verb!1.png! a \verb!8.png!): sono tutte
$512 \times 512$, in scala di grigi e contenenti dai due ai sei oggetti
geometrici a tinta unita su sfondo nero. Oltre a queste, vengono usate due foto
reali, sempre in scala di grigio ma di risoluzione $2048 \times 2048$:
\verb!A.png! e \verb!B.png!.

\subsection{Corruzione dell'insieme dei dati}
Per ogni immagine $x$ fra quelle sopra indicate, viene dapprima applicata una
sfocatura di matrice associata $A$ e poi un rumore uniforme di matrice associata
$\eta$ (entrambi costruiti a partire da una gaussiana) al fine di ottenere
un'immagine degradata $b$:
\begin{equation}
  b = Ax + \eta
\end{equation}
Tali operazioni sono ripetute più volte, ciascuna con valori diversi assegnati
ai parametri in questione (si veda \ref{results}).

\subsection{Risoluzione ingenua}
Una semplice ricostruzione $x*$ viene implementata a partire da:
\begin{equation}\label{eq:naive}
  x* = \arg \min_{x} \frac{1}{2}||Ax - b||^2_2
\end{equation}

\subsection{Risoluzione usando un termine di regolarizzazione di Tikhonov}
Introducendo un termine di regolarizzazione di Tikhonov, la \ref{eq:naive}
diventa:
\begin{equation}
  x* = \arg \min_{x} \frac{1}{2}||Ax - b||^2_2 + \frac{\lambda}{2}||x||^2_2
\end{equation}

\subsection{Risoluzione usando la variazione totale}
Usando invece la variazione totale come termine di fattorizzazione, la
\ref{eq:naive} diventa:
\begin{equation}
  x* = \arg \min_{x} \frac{1}{2}||Ax - b||^2_2 +
  \lambda\sum_i^n\sum_j^m\sqrt{||\nabla x(i,j)||_2^2+\epsilon^2}
\end{equation}

\section{Risultati} \label{results}

\subsection{Confronto}

Lorem ipsum dolor sit amet, consectetur adipiscing elit. Ut eget elit dolor. Aenean sodales mi a aliquet accumsan. Mauris molestie ultrices velit quis accumsan. Maecenas tortor neque, commodo et sagittis sed, consectetur vel diam. Maecenas venenatis sem magna, sollicitudin placerat purus laoreet a. Nullam convallis felis vitae ligula ultrices, in vestibulum augue auctor. Suspendisse risus purus, euismod ac ultricies eget, blandit et ex. Fusce nunc dui, feugiat ut velit at, dignissim consectetur sem. Quisque aliquet suscipit consequat. Sed in libero lectus. Proin tempor malesuada ultricies. Morbi lobortis rhoncus lorem et dignissim. Ut elementum diam at leo efficitur tempus. Curabitur vitae nibh auctor, ultricies mauris vel, faucibus neque. Quisque tincidunt orci condimentum molestie hendrerit. Donec et mattis nisi.
\begin{figure}
   \begin{center}
       \vspace*{-1.1in}
       \centerline{\scalebox{1.3}{\input{methods.pgf}}}
   \end{center}
   \vspace*{-0.2in}
   \caption{bla bla bla}
\end{figure}
Phasellus sit amet libero eu eros convallis blandit ac a mauris. Ut volutpat purus sed mi ullamcorper, sit amet porttitor nibh finibus. Aenean sed orci consequat, iaculis est a, auctor lectus. Maecenas ut volutpat eros. Curabitur volutpat felis sit amet lorem porta cursus. Duis lobortis sagittis elit. Mauris nec vulputate dolor, non dignissim tellus. Nam ultrices quam ac mauris cursus malesuada.

\subsection{Variazione dei parametri}

\begin{figure}
    \begin{center}
        \scalebox{0.65}{\input{vars-blur.tex}}
    \end{center}
    \begin{center}
        \scalebox{0.65}{\input{vars-noise.tex}}
    \end{center}
    \begin{center}
      \scalebox{0.65}{\input{vars-lambda.tex}}
    \end{center}
  \caption{bla bla bla}
\end{figure}

\subsection{Dati aggregati}

\begin{figure}
    \begin{center}
        \scalebox{0.65}{\input{aggregations.tex}}
    \end{center}
    \caption{dati aggregati}
\end{figure}

\subsection{Esecuzioni}

Nullam tempor eget sapien eget suscipit. Fusce bibendum leo odio, eget tempus erat volutpat ac. Vestibulum non elit mauris. Mauris ut aliquam lectus, sed tincidunt diam. Aenean vehicula velit id nisl rutrum, ac fermentum magna accumsan. Nulla eget est viverra, maximus lorem vel, mattis est. Nam id sapien leo.
\begin{figure}
    \begin{center}
        \scalebox{0.65}{\input{iterations-error.pgf}}
    \end{center}
    \caption{error}
\end{figure}
Etiam sit amet urna eu quam auctor facilisis et tristique nisl. Quisque varius turpis tincidunt, rhoncus tellus sed, bibendum velit. Suspendisse pellentesque venenatis sapien, ac blandit sem congue et. Ut accumsan scelerisque sapien, sit amet consequat nisl efficitur vitae. Nulla sed sem elit. Aenean vestibulum nisl in lacus pharetra, nec sollicitudin arcu tincidunt. Sed in aliquet mauris, facilisis viverra tellus. Nam ultrices ultrices magna, et rhoncus massa tempor sit amet. Mauris efficitur nunc sit amet quam molestie, quis aliquam ex mollis. Ut in sem rutrum, tempus eros et, volutpat libero. Praesent eget tincidunt sem. Suspendisse vitae nunc pellentesque, dapibus augue eu, tempus lacus. Praesent quis lectus purus. Nullam quis nulla odio. Quisque at felis in nulla placerat facilisis.
\begin{figure}
    \begin{center}
        \scalebox{0.65}{\input{iterations-objective.pgf}}
    \end{center}
    \caption{objective}
\end{figure}
In sapien turpis, ullamcorper tincidunt enim finibus, bibendum tempor quam. Suspendisse commodo ex ac massa aliquet, ac blandit erat rhoncus. Nulla porta massa quis egestas pretium. Sed aliquam magna augue, et fermentum sapien facilisis quis. Duis leo ex, pretium sed feugiat ac, viverra id elit. Ut cursus bibendum rutrum. Praesent tempor mattis ex, vel tempor mi pharetra in.
\begin{figure}
    \begin{center}
        \scalebox{0.65}{\input{iterations-gradient.pgf}}
    \end{center}
    \caption{gradient}
\end{figure}

\section{Conclusioni}

\end{document}
