\documentclass[a4paper]{article}
\usepackage[italian]{babel}
\usepackage{epigraph}
\usepackage{hhline}
% plotting
\usepackage{pgf}
\usepackage{xcolor}

\definecolor{bg}{RGB}{255,249,227}

\title{Ricostruzione di immagini \\ \large Relazione del progetto per l'insegnamento di Calcolo numerico}
\author{
  R. Gianmaria,
  L. Tagliavini,
  S. Volpe
}

\date{
	Universit\`a di Bologna \\
  \today
}

\begin{document}

\maketitle
\thispagestyle{empty}

\epigraph{Lo scopo del calcolo è la conoscenza, non i numeri.}
{\textit{R. W. Hamming}}
\epigraph{... ma per lo studente, i numeri sono spesso la via migliore verso la
conoscenza}
{\textit{A. Ralston}}

\section{Introduzione}

\section{Strumenti}

\section{Procedimento}

\section{Risultati}

\subsection{Confronto}

\begin{figure}
   \begin{center}
       \vspace*{-1.1in}
       \centerline{\scalebox{1.3}{\input{methods.pgf}}}
   \end{center}
   \vspace*{-0.2in}
   \caption{bla bla bla}
\end{figure}

\subsection{Variazione dei parametri}

\begin{figure}
    \begin{center}
        \scalebox{0.65}{\input{vars-blur.tex}}
    \end{center}
    \begin{center}
        \scalebox{0.65}{\input{vars-noise.tex}}
    \end{center}
    \begin{center}
      \scalebox{0.65}{\input{vars-lambda.tex}}
    \end{center}
    \caption{bla bla bla}
\end{figure}

\subsection{Dati aggregati}

\begin{figure}
    \begin{center}
        \scalebox{0.65}{\input{aggregations.tex}}
    \end{center}
    \caption{dati aggregati}
\end{figure}

\subsection{Esecuzioni}

\begin{figure}
    \begin{center}
        \scalebox{0.65}{\input{iterations-error.pgf}}
    \end{center}
    \caption{error}
\end{figure}

\begin{figure}
    \begin{center}
        \scalebox{0.65}{\input{iterations-objective.pgf}}
    \end{center}
    \caption{objective}
\end{figure}

\begin{figure}
    \begin{center}
        \scalebox{0.65}{\input{iterations-gradient.pgf}}
    \end{center}
    \caption{gradient}
\end{figure}

\section{Conclusioni}

\end{document}
