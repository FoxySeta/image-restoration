\documentclass[a4paper]{article}
\usepackage[italian]{babel}
\usepackage{epigraph}
\usepackage{hhline}
% plotting
\usepackage{pgf}
\usepackage{xcolor}

\definecolor{bg}{RGB}{255,249,227}

\title{
  Ricostruzione di immagini \\ \large Relazione del progetto per
  l'insegnamento di Calcolo numerico
}
\author{
  R. Gianmaria,
  L. Tagliavini,
  S. Volpe
}

\date{
	Universit\`a di Bologna \\
  \today
}

\begin{document}

\maketitle
\thispagestyle{empty}

\pagebreak
\tableofcontents
\pagebreak

\epigraph{Lo scopo del calcolo è la conoscenza, non i numeri.}
{\textit{R. W. Hamming}}
\epigraph{... ma per lo studente, i numeri sono spesso la via migliore verso la
conoscenza}
{\textit{A. Ralston}}

\section{Introduzione}
La rappresentabilità delle immagini digitali in forma matriciale fa sì che la
loro manipolazione possa trarre beneficio dagli strumenti del calcolo numerico:
un problema di interesse pratico è la ricostruzione di tali immagini quando
degradate. Nell'esperimento che segue, ci proponiamo di effettuare un'analisi
compararata di esempi di implementazioni dei metodi risolutivi più noti per la
risoluzione di questo tipo di problemi. Assumeremo, fra le conoscenze pregresse
della lettrice e del lettore, il modello di formazione-registrazione di
un'immagine digitale e i principali metodi di regolarizzazione.

\section{Strumenti}
Tutte le procedure necessarie all'esperimento sono state implementate in Python
3. In particolare, viene fatto uso dei pacchetti:
\begin{itemize}
  \item \verb!matplotlib! per la creazione di grafici;
  \item \verb!numpy! per l'uso di matrici;
  \item \verb!scipy! per risolvere i problemi di ottimizzazione;
  \item \verb!skimage! per il calcolo delle metriche delle immagini.
\end{itemize}

\section{Procedimento}

\subsection{Generazione dell'insieme dei dati}
Sono state preparate otto immagini (da \verb!1.png! a \verb!8.png!): sono tutte
$512 \times 512$, in scala di grigi e contenenti dai due ai sei oggetti
geometrici a tinta unita su sfondo nero. Oltre a queste, vengono usate due foto
reali, sempre in scala di grigio ma di risoluzione $2048 \times 2048$:
\verb!A.png! e \verb!B.png!.

\subsection{Corruzione dell'insieme dei dati}
Per ogni immagine $x$ fra quelle sopra indicate, viene dapprima applicata una
sfocatura di matrice associata $A$ e poi un rumore uniforme di matrice associata
$\eta$ (entrambi costruiti a partire da una gaussiana) al fine di ottenere
un'immagine degradata $b$:
\begin{equation}
  b = Ax + \eta
\end{equation}
Tali operazioni sono ripetute più volte, ciascuna con valori diversi assegnati
ai parametri in questione (si veda \ref{results}).

\subsection{Risoluzione ingenua}
Una semplice ricostruzione $x*$ viene implementata a partire da:
\begin{equation}\label{eq:naive}
  x* = \arg \min_{x} \frac{1}{2}||Ax - b||^2_2
\end{equation}

\subsection{Risoluzione usando un termine di regolarizzazione di Tikhonov}
Introducendo un termine di regolarizzazione di Tikhonov, la \ref{eq:naive}
diventa:
\begin{equation}
  x* = \arg \min_{x} \frac{1}{2}||Ax - b||^2_2 + \frac{\lambda}{2}||x||^2_2
\end{equation}

\subsection{Risoluzione usando la variazione totale}
Usando invece la variazione totale come termine di fattorizzazione, la
\ref{eq:naive} diventa:
\begin{equation}
  x* = \arg \min_{x} \frac{1}{2}||Ax - b||^2_2 +
  \lambda\sum_i^n\sum_j^m\sqrt{||\nabla x(i,j)||_2^2+\epsilon^2}
\end{equation}

\section{Risultati} \label{results}
Dove non specificato, nei risultati che seguono viene usato il metodo dei
gradienti coniugati (GC) anziché quello del gradiente (G). 

\subsection{Confronto}
In figura \ref{fig:comparison} sono riportati i risultati dei diversi metodi di
ricostruzione su una immagine geometrica e due fotografie.
\begin{figure}\label{fig:comparison}
   \begin{center}
       \vspace*{-1.1in}
       \centerline{\scalebox{1.3}{\input{methods.pgf}}}
   \end{center}
   \vspace*{-0.2in}
   \caption{confronto dei risultati con parametro di regolarizzazione (dove
   usato) pari a $0.04$. Gli indici di PSNR e MSE non si riferiscono ad alcuna
   immagine presente, bensì alle degradazioni.}
\end{figure}
Si osserva che la differenza fra diverse versioni di una stessa immagine è molto
più evidente in \verb!1.png! che negli altri casi: qui, la ricostruzione più
"rumorosa", nonché di peggior qualità, è visibilmente quella naïve, mentre la
ricostruzione con variazione totale è quella che offre effetti più regolari.

\subsection{Variazione dei parametri}
Variando i parametri in gioco, è possibile tabulare PSNR e MSE ottenuti con i
vari metodi: il risultato di questa operazione è mostrato in figura
\ref{fig:parameters}.
Confrontando celle diverse, ci si accorge che a valori del PSNR più alti
corrispondono sempre valori dell'MSE più bassi e viceversa: ci limitamo quindi
all'osservazione del primo di questi due parametri. A parità di parametri, esso
cresce all'aumentare dell'indice della colonna: in alcune applicazioni del
metodo naïve vengono addirittura raggiunti valori negativi. All'aumentare
dell'intensità del rumore, i valori scendono, mentre nel caso del parametro di
regolarizzazione (fatta eccezione per la prima colonna), l'andamento del PSNR
sembra avere un atteggiamento concavo.
\begin{figure}\label{fig:parameters}
    \begin{center}
        \scalebox{0.65}{\input{vars-blur.tex}}
    \end{center}
    \begin{center}
        \scalebox{0.65}{\input{vars-noise.tex}}
    \end{center}
    \begin{center}
      \scalebox{0.65}{\input{vars-lambda.tex}}
    \end{center}
  \caption{antologia di tabelle che confrontano le coppie ordinate (PSNR, MSE)
  di varie ricostruzioni della prima immagine di prova al variare della
  sfocatura, dell'intensità del rumore e del parametro di regolarizzazione
  usato.}
\end{figure}

\subsection{Dati aggregati}
La tabella \ref{fig:aggregation} riassume PSNR e MSE dell'intero insieme di
immagini geometriche tramite media e deviazione standard.
\begin{figure}\label{fig:aggregation}
    \begin{center}
        \scalebox{0.65}{\input{aggregations.tex}}
    \end{center}
    \caption{medie e deviazioni standard delle coppie ordinate (PSNR, MSE) al
    variare dei parametri. Il campione preso in esame è composto dalle
    ricostruzioni effettuate su ciascuna immagine geomatrica dell'insieme di
    dati. Il metodo usato è quello dei gradienti coniugati, mentre la variazione totale è stata scelta come
    termine di regolarizzazione.}
\end{figure}
Si osserva che la deviazione standard del PSNR si minimizza per valori bassi
del rumore dell'immagine degradata e del parametro di regolarizzazione usato.

\subsection{Esecuzioni}
Concludiamo confrontando, iterazione per iterazione, le esecuzioni del metodo
del gradiente e di quello dei gradienti coniugati su \verb!1.png!; il termine di
regolarizzazione usato è quello di Tikhonov.
\begin{figure}
    \begin{center}
        \scalebox{0.65}{\input{iterations-error.pgf}}
    \end{center}
    \caption{MSE dei due metodi a confronto}
\end{figure}
\begin{figure}
    \begin{center}
        \scalebox{0.65}{\input{iterations-objective.pgf}}
    \end{center}
    \caption{valori della funzione obiettivo da minimizzare a confronto}
\end{figure}
\begin{figure}
    \begin{center}
        \scalebox{0.65}{\input{iterations-gradient.pgf}}
    \end{center}
    \caption{norme dei gradienti della funzione obiettivo da minimizzare a
    confronto}
\end{figure}
In tutti i casi le funzioni convergono dall'alto a un valore basso. Nel caso
dello studio del valore della funzione obiettivo e della norma del suo
gradiente, i valori a cui i due metodi convergono approssimativamente
coincidono. Nel caso dell'MSE e del valore della funzione obiettivo, i grafici
sono monotoni non crescenti per entrambi i metodi. In tutti i casi, il metodo
dei gradienti coniugati converge appena qualche iterazione prima del metodo del
gradiente.

\section{Conclusioni}

\end{document}
