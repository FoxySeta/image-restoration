\documentclass[a4paper]{article}
\usepackage[italian]{babel}
\usepackage{epigraph}
\usepackage{hhline}
% plotting
\usepackage{pgf}
\usepackage{xcolor}

\definecolor{bg}{RGB}{255,249,227}

\title{Ricostruzione di immagini \\ \large Relazione del progetto per l'insegnamento di Calcolo numerico}
\author{
  R. Gianmaria,
  L. Tagliavini,
  S. Volpe
}

\date{
	Universit\`a di Bologna \\
  \today
}

\begin{document}

\maketitle
\thispagestyle{empty}

\epigraph{Lo scopo del calcolo è la conoscenza, non i numeri.}
{\textit{R. W. Hamming}}
\epigraph{... ma per lo studente, i numeri sono spesso la via migliore verso la
conoscenza}
{\textit{A. Ralston}}

\section{Introduzione}

 \begin{figure}
     \begin{center}
         \vspace*{-1.1in}
         \centerline{\scalebox{1.3}{\input{methods.pgf}}}
     \end{center}
     \vspace*{-0.2in}
     \caption{bla bla bla}
 \end{figure}

\section{Strumenti}

\section{Procedimento}

% \begin{figure}
%     \begin{center}
%         \scalebox{0.65}{\input{iterations-error-CG.pgf}}
%         \scalebox{0.65}{\input{iterations-error-CG-img.pgf}}
%     \end{center}
%     \caption{error cg (scipy)}
% \end{figure}
% \begin{figure}
%     \begin{center}
%         \scalebox{0.65}{\input{iterations-error-G.pgf}}
%         \scalebox{0.65}{\input{iterations-error-G-img.pgf}}
%     \end{center}
%     \caption{error g (our)}
% \end{figure}

% \begin{figure}
%     \begin{center}
%         \scalebox{0.65}{\input{iterations-objective-CG.pgf}}
%         \scalebox{0.65}{\input{iterations-objective-G.pgf}}
%     \end{center}
%     \caption{objetive cg}
% \end{figure}
%
% \begin{figure}
%     \begin{center}
%         \scalebox{0.65}{\input{iterations-gradient-CG.pgf}}
%         \scalebox{0.65}{\input{iterations-gradient-G.pgf}}
%     \end{center}
%     \caption{gradient cg}
% \end{figure}

\section{Risultati}

\section{Conclusioni}

\end{document}
